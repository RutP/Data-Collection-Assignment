\documentclass{article}
\usepackage[utf8]{inputenc}

\title{HIV\_Awareness}
\author{rutabingwa peter }
\date{May 2017}

\begin{document}


\maketitle


\section{Introduction}
HIV disease continues to be a serious health issue for parts of the world. Worldwide, there were about 2.1 million new cases of HIV in 2015. About 36.7 million people are living with HIV around the world, and as of June 2016, 17 million people living with HIV were receiving medicines to treat HIV, called antiretroviral therapy (ART). 

\section{Problem}
An estimated 1.1 million people died from AIDS-related illnesses in 2015. Sub-Saharan Africa, which bears the heaviest burden of HIV/AIDS worldwide, accounts for 65\% of all new HIV infections. Other regions significantly affected by HIV/AIDS include Asia and the Pacific, Latin America and the Caribbean, and Eastern Europe and Central Asia. We neglect the fact that some people completely don’t know its existence or know little about the dangerous prob
We cannot expect to prevent something we have no control over. If our people do not understand what is there to understand about HID/AIDs, then we are in for a lot to handle (Danger). Without a proper laid out plan to cover such unsolved issues, we are in for ton of infected individuals. A very ample example would me in my country, Uganda and deep in rural areas such as kisoro, Few individuals have had sexual education and the other numbers are completely not immune to the effects of its ignorance. 
\section{HIV\_Awareness Form}
This comes in handy to generate stats to establish whether everyone really knows and understand the HIV/AIDs concept. It is designed in such a way that the individual includes their Name, age, Location and also include a Picture of oneself. And following all that are questions that examine ones awareness. It is important villagers are informed of the danger of HIV-AID and how to avoid it. This can only be achieved after knowing their point of view regarding this aspect.

 
\section{Conclusion}

The forms help assess the knowledge people have on this HIV awareness idea. After a deep survey and firm conclusions put down. 
the participants are subjected to serious sensitization about the matter and see to eradicate the problem completely. With this in place, the rate of acquiring this virus is reduced.


\end{document}
